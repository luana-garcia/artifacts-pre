\section{Conclusion}
This work explored the application of the Anchors XAI method across two distinct domains: tabular data for fairness auditing and high-dimensional spatial data for weather forecasting. The results demonstrate that Anchors possesses a unique capability to provide focused, direct, and actionable insights, regardless of the data's complexity.

In the tabular domain with the Folktables dataset, Anchors proved to be a powerful tool for identifying specific vulnerabilities and biases within model predictions. By generating high-precision rules, it pinpointed the exact demographic and socioeconomic factors that led to discriminatory outcomes, such as systematically lower income predictions for certain groups. This granular understanding is the first crucial step toward designing targeted bias mitigation strategies, moving beyond simply detecting bias to explaining its precise mechanisms.

When applied to the high-dimensionality of meteorological data, Anchors revealed its versatility. The method transitioned from auditing model fairness to auditing model reasoning in a physical context. It successfully identified not just if a prediction was made, but which specific areas and meteorological variables were most influential for that prediction. This transforms Anchors into a powerful tool for regional influence analysis, allowing meteorologists to understand which upstream regions have the greatest impact on a local forecast.

Furthermore, the method demonstrated potential for proactive vulnerability mapping. By highlighting surrounding areas that significantly influence a target location, Anchors can reveal geographic points of failure or sensitivity. This is invaluable for planning and preparedness, as it allows us to identify zones where extreme weather events elsewhere could have a critical downstream impact, thereby improving risk assessment and resource allocation.

In conclusion, from auditing societal biases in algorithms to diagnosing the physical drivers of weather models, the Anchors method has proven to be exceptionally adept at providing focused explanations. It moves past global model interpretation to deliver local, instance-based insights that are directly applicable to solving real-world problems, whether they pertain to social justice or natural phenomena. This work establishes a robust foundation for using explainability not just as a diagnostic tool, but as a guide for actionable intervention in both technical and societal systems.