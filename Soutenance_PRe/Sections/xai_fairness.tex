\subjectPresentation{2}{XAI for the USA Census}{color2}

\subjectDevelopment{Analyse Fairness on the USA Census}{
	\begin{itemize}
		\item Can we find bias in the ML models?
		\item If so, can we use XAI to identify where's the problem?
	\end{itemize}
	\vfill
	\centering
	We used the 'SEX' as the sensible variable.
}{color2}

\subjectDevelopment{Trained Models}{
\begin{itemize}
	\item \textbf{Logistic Regression}
	
    \item \textbf{XGBoost} (eXtreme Gradient Boosting)
	
	\item \textbf{Hist Gradient Boosting} (Skrub's Scikit-learn implementation)
	
	\item \textbf{Simple Neural Network}
\end{itemize}
}{color2}

\subjectDevelopment{Fairness Metrics}{
\begin{itemize}
	\item \textbf{Accuracy}: The proportion of correct predictions (both true positives and true negatives) among all predictions.

	\item \textbf{Disparate Impact (DI)}: Measures the ratio between the \textbf{proportion of positive outcomes for the protected group (women) versus the privileged group (men)}. Values close to 1 indicate fairness, while values below 1 suggest bias against the protected group.

	\item \textbf{Equality of Odds}: Examines whether both groups have equal true positive rates and equal false positive rates. Values closer to 1 indicate better fairness.

	\item \textbf{Sufficiency}: Assesses whether the probability of the true outcome is the same across groups given the predicted outcome. Values closer to 1 indicate better fairness.
\end{itemize}
}{color2}

\subjectDevelopment{Performance and Fairness metrics}{
\begin{table}[h]
\centering
\caption{Model Performance Comparison Across States (Accuracy and Fairness Metrics)}
\label{tab:folktables-results}
\resizebox{\textwidth}{!}
{
\begin{tabular}{llcccccc}
\toprule
\textbf{Model} & \textbf{Training} & \textbf{Testing} & \textbf{Accuracy} & \textbf{Disparate Impact} & \textbf{Equality of Odds} & \textbf{Sufficiency} \\
\midrule
\rowcolor{gray!10}
\multirow{6}{*}{Logistic Regression} 
& \multirow{2}{*}{CA} & CA & 0.56 & 0.67 & 0.84 & 0.95 \\
& & USA & 0.52 & 0.66 & 0.88 & 0.86 \\
\cmidrule(lr){2-7}
& \multirow{2}{*}{TX} & TX & 0.52 & 0.46 & 0.65 & 0.90 \\
& & USA & 0.51 & 0.46 & 0.67 & 0.95 \\
\cmidrule(lr){2-7}
& \multirow{2}{*}{NY} & NY & 0.51 & 0.66 & 0.82 & 0.93 \\
& & USA & 0.52 & 0.64 & 0.86 & 0.88 \\
\midrule

\rowcolor{gray!10}
\multirow{6}{*}{XGBoost} 
& \multirow{2}{*}{CA} & CA & 0.64 & 0.72 & 0.91 & 0.96 \\
& & USA & 0.58 & 0.67 & 0.92 & 0.90 \\
\cmidrule(lr){2-7}
& \multirow{2}{*}{TX} & TX & 0.60 & 0.58 & 0.83 & 0.93 \\
& & USA & 0.58 & 0.58 & 0.85 & 0.95 \\
\cmidrule(lr){2-7}
& \multirow{2}{*}{NY} & NY & 0.61 & 0.75 & 0.92 & 0.96 \\
& & USA & 0.57 & 0.67 & 0.92 & 0.90 \\
\midrule

\rowcolor{gray!10}
\multirow{6}{*}{HistGradientBoosting} 
& \multirow{2}{*}{CA} & CA & 0.63 & 0.71 & 0.90 & 0.94 \\
& & USA & 0.58 & 0.67 & 0.92 & 0.90 \\
\cmidrule(lr){2-7}
& \multirow{2}{*}{TX} & TX & 0.61 & 0.54 & 0.78 & 0.96 \\
& & USA & 0.58 & 0.56 & 0.83 & 0.97 \\
\cmidrule(lr){2-7}
& \multirow{2}{*}{NY} & NY & 0.60 & 0.68 & 0.89 & 0.97 \\
& & USA & 0.58 & 0.64 & 0.90 & 0.91 \\
\midrule

\rowcolor{gray!10}
\multirow{6}{*}{Neural Network} 
& \multirow{2}{*}{CA} & CA & 0.52 & 0.88 & 1.03 & 0.85 \\
& & USA & 0.46 & 0.65 & 0.86 & 0.86 \\
\cmidrule(lr){2-7}
& \multirow{2}{*}{TX} & TX & 0.50 & 0.61 & 0.87 & 0.94 \\
& & USA & 0.49 & 0.77 & 0.96 & 0.81 \\
\cmidrule(lr){2-7}
& \multirow{2}{*}{NY} & NY & 0.51 & 0.76 & 0.93 & 0.92 \\
& & USA & 0.48 & 0.66 & 0.88 & 0.87 \\
\bottomrule
\end{tabular}%
}
\end{table}
}{color2}

\subjectDevelopment{Applying Anchors}{
Our focus is on Meteo today, so it's just important to know that our results found zones in the population with high indication of unfairness.
}{color2}